\input{preamble}

% Pour bac.
\newcommand{\BAC}[1]{#1}
\newcommand{\MAITRISE}[1]{}
% Pour maitrise
\renewcommand{\BAC}[1]{}
\renewcommand{\MAITRISE}{}


\author[Guy Tremblay, D\'epartement d'informatique]
{Guy Tremblay\\Professeur}

\institute{%
D\'epartement d'informatique\\
%
UQAM\\%
\url{http://www.labunix.uqam.ca/~tremblay}%
}



\date{$1^{er}$ octobre 2015}

\begin{document}

\title{Assemblage de logiciels}


%------------------------------------------------
\begin{frame}
\frametitle{MGL7460 --- Automne 2015}

  \titlepage

\NOTE{\ }
\end{frame}

%------------------------------------------------
\begin{frame}
\frametitle{Contenu}

\large

\tableofcontents


\NOTE{\ }

\end{frame}

%\SEULEMENT{rake}



%SSSSSSSSSSSSSSSSSSSSSSSSSSSSSSSSSSSSSSSSSSSSSSSSSSSSSSSSSSSSS
\section{Introduction/motivation}
%SSSSSSSSSSSSSSSSSSSSSSSSSSSSSSSSSSSSSSSSSSSSSSSSSSSSSSSSSSSSS

%------------------------------------------------
\begin{frame}
\frametitle{\AlertTitle{Rappel}~: Parmi les premi\`eres choses \`a faire, quand on
d\'eveloppe du code de fa\c{c}on professionnelle\ldots}

<<{\em [Y]ou need to get the development infrastructure environment in
order.  That means adopting (or improving) the fundamental Starter Kit
practices:
\begin{itemize}
\item {Version control}

\item {Unit Testing}

\item \alert{Build automation}
\end{itemize}

{Version control} needs to come before anything else. It's the first
bit of infrastructure we set up on any project.}>>

\bigskip


{\small<<Practices of an Agile Developer---Working in the Real
World>>, Subramaniam \& Hunt, 2006.}


\NOTE{\ }

\end{frame}

%------------------------------------------------
\begin{frame}
\frametitle{\AlertTitle{Rappel}~: Quelques \emph{tips} de Hunt \& Thomas}

\begin{itemize}

\item[11.] \alert{DRY---Don't Repeat Yourself}

Every piece of knowledge must have a single, unambiguous,
authoritative representation within a system.

\vfill

\item[21.] \alert{Use the Power of Command Shells}

Use the shell when graphical user interfaces don’t cut it.

\vfill
\item[61.] \alert{Don't Use Manual Procedures}

A shell script or batch file will execute the same instructions, in
the same order, time after time.

\VF

\end{itemize}

\NOTEvide


\end{frame}


%------------------------------------------------
\begin{frame}
\frametitle{Bref, l'\'etape de \TT{build} est cruciale}
\framesubtitle{Source~: \url{geek-and-poke.com}}

\begin{tabular}{p{3in}p{2in}}
\hspace*{-1.3cm}\includegraphics[width=3.1in]{Figures/geek-poke-build1}
&
\hspace*{-1.8cm}\includegraphics[width=2.2in]{Figures/geek-poke-build2}
\end{tabular}



\NOTE{\ }

\end{frame}

%SSSSSSSSSSSSSSSSSSSSSSSSSSSSSSSSSSSSSSSSSSSSSSSSSSSSSSSSSSSSS
\section{Qu'est-ce que l'assemblage de logiciels?}
%SSSSSSSSSSSSSSSSSSSSSSSSSSSSSSSSSSSSSSSSSSSSSSSSSSSSSSSSSSSSS

%------------------------------------------------
\begin{frame}
\frametitle{Qu'est-ce que l'assemblage d'un logiciel?}

\begin{quote}
Historically, \alert{build} has often referred either to the process
of \alert{converting source code files into standalone software artifact(s)}
that can be run on a computer, or the result of doing so.

\bigskip

The process of building a computer program is usually managed by a
\alert{build tool}, a program that coordinates and controls other
programs. [\ldots] \alert{The build utility needs to compile and link the
various files, \underline{in the correct order}}. If the source code in a
particular file has not changed then it may not need to be
recompiled~[\ldots].
\end{quote}

\source{\url{https://en.wikipedia.org/wiki/Software_build}}


\NOTE{\ }

\end{frame}

%------------------------------------------------
\begin{frame}
\frametitle{Il faut distinguer entre \emph{composants sources} et
\emph{composants d\'eriv\'es}}

\begin{block}{Composants sources (\emph{source components})}
Les composants qui sont --- et doivent \^etre --- cr\'e\'es
\alert{manuellement} par les d\'eveloppeurs.
\end{block}


\bigskip

\only<1>{\ \ }
\only<2>{Exemple pour programmes C~: Fichiers \TT{*.[hc]}, \TT{Makefile}.}

\VF

\begin{block}{Composants d\'eriv\'es (\emph{derived components})}
Les composants qui peuvent \^etre \'et\'e cr\'e\'es \alert{automatiquement} par
la machine, sans l'intervention explicite des d\'eveloppeurs.

\end{block}

\bigskip

\only<1>{\ \ \\\ \ } \only<2>{Exemple pour programmes C~: Fichiers
\TT{*.o}, ex\'ecutables.}

\VF

\source{<<\emph{Essential Open Source Toolset}>>, Zeller \& Krinke}
\NOTE{\ }

\end{frame}

%------------------------------------------------
\begin{frame}
\frametitle{Il faut distinguer entre \emph{composants sources} et
\emph{composants d\'eriv\'es}}

\alert{Remarque importante en lien avec le contr\^ole du
code source~:} 

\begin{itemize}
\bigskip

\item Seuls les fichiers pour les composants \alert{sources} doivent
\^etre mis dans le syst\`eme de contr\^ole du code source.

\bigskip

\Implique\ Les fichiers pour les composants d\'eriv\'es \alert{ne devraient pas
\^etre mis} dans le syst\`eme de contr\^ole du code source.  

\end{itemize}
\VF

\NOTEvide

\end{frame}

%------------------------------------------------
\begin{frame}
\frametitle{Le r\^ole d'un outil d'assemblage de logiciels}

Le r\^ole d'un outil d'assemblagee de logiciels est d'assurer qu'un
logiciel soit assembl\'e\ldots

\VF

\begin{itemize}
\item \alert{\`a partir des bons composants sources}

\VF

\item \alert{toujours de la bonne fa\c{c}on}

\VF

\item \alert{sans intervention humaine}

\VF

\item \alert{rapidement} --- par ex., en reg\'en\'erant
le nombre \alert{minimal} de composants d\'eriv\'es \Implique\ on ne
recompile que le strict n\'ecessaire


\VF

\end{itemize}

\NOTE{\ }

\end{frame}

%------------------------------------------------
\begin{frame}
\frametitle{Tous les outils d'assemblage reposent sur un mod\`ele du logiciel \`a
construire\Implique\ \AlertTitle{Graphe des d\'ependances}}

\begin{center}
\includegraphics[width=11cm]{Figures/graphe-dependances}

\end{center}

\NOTE{
L\'egende~:
\begin{Items}
\item Boite avec lignes plus \'epaisses~: composants sources 
\item Boite avec lignes plus minces~: composants d\'eriv\'es 
\end{Items}
}

\end{frame}

%------------------------------------------------
\begin{frame}
\frametitle{Le processus d'assemblage repose sur l'analyse du graphe
de d\'ependances}

Soit un composant d\'eriv\'e $A$ qui d\'epend de $A_1$, $A_2$, \ldots,
$A_n$.


\bigskip

Alors, le composant \alert{$A$ doit \^etre reg\'en\'er\'e} si une des
conditions suivantes s'applique~:
\begin{enumerate}
\item $A$ n'existe pas

\item un des composants $A_i$ a \'et\'e modifi\'e

\item un des composants $A_i$ doit \^etre reg\'en\'er\'e

\end{enumerate}

\VF

\alert{Donc~: } il s'agit donc d'un processus \alert{r\'ecursif}!

\source{<<\emph{Essential Open Source Toolset}>>, Zeller \& Krinke}

\NOTE{\ }

\end{frame}

\section{Quelques outils d'assemblage de logiciels}

%SSSSSSSSSSSSSSSSSSSSSSSSSSSSSSSSSSSSSSSSSSSSSSSSSSSSSSSSSSSSS
\Subsection{L'outil \TT{make} sur Unix}
%SSSSSSSSSSSSSSSSSSSSSSSSSSSSSSSSSSSSSSSSSSSSSSSSSSSSSSSSSSSSS

\input{build_outil-make}


%SSSSSSSSSSSSSSSSSSSSSSSSSSSSSSSSSSSSSSSSSSSSSSSSSSSSSSSSSSSSS
\Subsection{L'outil \TT{Ant}}
%SSSSSSSSSSSSSSSSSSSSSSSSSSSSSSSSSSSSSSSSSSSSSSSSSSSSSSSSSSSSS

\input{build_outil-ant}

%SSSSSSSSSSSSSSSSSSSSSSSSSSSSSSSSSSSSSSSSSSSSSSSSSSSSSSSSSSSSS
\Subsection{L'outil \TT{Maven}}
%SSSSSSSSSSSSSSSSSSSSSSSSSSSSSSSSSSSSSSSSSSSSSSSSSSSSSSSSSSSSS

\input{build_outil-maven}


%SSSSSSSSSSSSSSSSSSSSSSSSSSSSSSSSSSSSSSSSSSSSSSSSSSSSSSSSSSSSS
\Subsection{L'outil \TT{Rake}}
%SSSSSSSSSSSSSSSSSSSSSSSSSSSSSSSSSSSSSSSSSSSSSSSSSSSSSSSSSSSSS

\input{build_outil-rake}

%SSSSSSSSSSSSSSSSSSSSSSSSSSSSSSSSSSSSSSSSSSSSSSSSSSSSSSSSSSSSS
\section{Conclusion}
%SSSSSSSSSSSSSSSSSSSSSSSSSSSSSSSSSSSSSSSSSSSSSSSSSSSSSSSSSSSSS

%------------------------------------------------
\begin{frame}
\frametitle{Tableau synth\'etique comparant \TT{make}, \TT{ANT} et \TT{Maven}}

\hspace*{-0.95cm}\includegraphics[width=13cm]{Figures/make-ant-maven}

\VF

\Source{<<The Evolution of {Java} Build Systems>>, McIntosh et al.}


\NOTE{\ }

\end{frame}

%------------------------------------------------
\begin{frame}
\frametitle{La maintenance des scripts d'assemblage peut
demander un effort non-n\'egligeable \frownie}

Certaines \'etudes ont montr\'e que, en moyenne, la conception et
l'entretien du syst\`eme de \emph{build} pouvaient entrainer des
\alert{surco\^uts} (\emph{overhead}) de l'ordre de 10~\%.

\pause

\VF

\begin{itemize}
\item Ajouter de nouvelles r\`egles pour traiter des nouveaux fichiers
sources

\item Ajuster la configuration du compilateur ou de l'\'editeur de
liens

\item \ldots
\end{itemize}

\VF

\Source{Article cit\'e dans <<The Evolution of {Java} Build Systems>>,
McIntosh et al.}

\VF
\NOTE{\ }

\end{frame}


%------------------------------------------------
\begin{frame}
\frametitle{Tous les outils d'assemblage utilisent un DSL}

\begin{block}{DSL = \emph{Domain Specific Language}}
\begin{quote}

A computer \alert{programming language} of
\Alert{orange}{limited expres\-siveness} focused on a
\Alert{magenta}{particular domain}.

\bigskip

\source{<<\emph{Domain-Specific Languages}>>, Fowler}
\end{quote}
\end{block}

\VF


\NOTE{\ }

\end{frame}

%------------------------------------------------
\begin{frame}
\frametitle{Tous les outils d'assemblage utilisent un DSL}

\begin{block}{DSL Externe}
\begin{quote}
<<An \alert{external DSL} is a completely separate language, for which
you [need] a full parser.>>
\end{quote}
\end{block}

\VF

\begin{block}{DSL Interne}
\begin{quote}
<<An \alert{internal DSL} is an idiomatic way of using a
general-purpose language.>>
\end{quote}

\end{block}


\VF


\NOTE{\ }

\end{frame}

%------------------------------------------------
\begin{frame}
\frametitle{Outils de \emph{build} et DSL}

\begin{center}
\begin{tabular}{||l|l|l||}\hline\hline
{\bf Outil} & {\bf Type de DSL} & {\bf Syntaxe}\\\hline\hline
Make & Externe &  \emph{Ad hoc}\\\hline
Ant & Externe &  XML\\\hline
Maven & Externe &  XML\\\hline
Rake & \alert{Interne} &  \alert{Ruby}\\\hline\hline
\end{tabular}
\end{center}

\NOTE{\ }

\end{frame}


%------------------------------------------------
\begin{frame}
\frametitle{R\'ef\'erences}

\nocite{HuntTho00}
\nocite{SubramaniamHun06}
\nocite{Rasmusson10}
\nocite{HumbleFar11}
\nocite{ZellerKri05}
\nocite{Koleshko14}
\nocite{Fowler11}
\nocite{McintoshHadHas12}

\bibliographystyle{alpha}

\vspace{-0.25cm}

{\tiny
\bibliography{%
biblio/design%
,%
biblio/varia%
,%
biblio/comp%
,%
biblio/arch+pp%
}
}

\NOTEvide


\end{frame}

\end{document}
