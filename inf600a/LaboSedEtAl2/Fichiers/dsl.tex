\input{preamble}

% Pour bac.
\newcommand{\BAC}[1]{#1}
\newcommand{\MAITRISE}[1]{}
% Pour maitrise
\renewcommand{\BAC}[1]{}
\renewcommand{\MAITRISE}{}

\author[Guy Tremblay, D\'epartement d'informatique]
{Guy Tremblay\\Professeur}

\institute{D\'epartement
d'informatique\\UQAM\\\url{http://www.labunix.uqam.ca/~tremblay}}

\date{22--29 octobre 2015}
\title{Langages sp\'ecifiques au domaine}


%%%%%%%%%%%%%%%%%%%%%%%%%%%%%%%%%%%%%%%%%%%%%%%%%%%%%%%%%%%%%%%%%

\begin{document}

%\SEULEMENT{dsl_exemple-documents-java}


%------------------------------------------------
\begin{frame}
\frametitle{MGL7460 --- Automne 2015}
\titlepage
\NOTE{\ }
\end{frame}

\begin{frame}
\frametitle{Contenu}
\small
\tableofcontents
\NOTE{\ }
\end{frame}


%%%%%%%%%%%%%%%%%%%%%%%%%%%%%%
\section{Introduction}

%------------------------------------------------
\begin{frame}[fragile]
\frametitle{Avez-vous d\'ej\`a utilis\'e\ldots\ \AlertTitle{make}?}


\begin{columns}

\begin{column}{5cm}
\begin{block}{}
{\footnotesize
\begin{semiverbatim}
  \alert{$ cat Makefile}
  hello: hello.c
      gcc -o hello hello.c
  
  clean:
      rm -f hello hello.o
\end{semiverbatim}
}
\end{block}
\end{column}


\begin{column}{5cm}
\begin{block}{}
{\footnotesize
\begin{semiverbatim}
\alert{$ make}
gcc -o hello hello.c

\end{semiverbatim}
}
\end{block}
\end{column}

\end{columns}

\NOTE{Used, on Unix, to automate program build tasks, i.e., to compile
and link programs but also, more generally, to automate tasks.}

\end{frame}

%------------------------------------------------
\begin{frame}[fragile]
\frametitle{Avez-vous d\'ej\`a utilis\'e\ldots\ \AlertTitle{SQL}?}

\begin{columns}
\begin{column}{4.2cm}
\begin{block}{}
{\footnotesize
\begin{semiverbatim}
SELECT title, price
  FROM  Book
   WHERE price > 30.00
  ORDER BY price;
\end{semiverbatim}
}
\end{block}
\end{column}


\begin{column}{6cm}
\begin{block}{}
{\footnotesize
\begin{semiverbatim}
Title                     Price
--------------                 -----
DSLs in Action            39.99
Domain-Specific Languages 45.99
Domain-Driven Design      49.99
\end{semiverbatim}
}
\end{block}
\end{column}

\end{columns}


\NOTE{SQL = Structured Query Language = language for managing data in
relational databases.}


\end{frame}



%------------------------------------------------
\begin{frame}[fragile]
\frametitle{Avez-vous d\'ej\`a utilis\'e\ldots\ \AlertTitle{HTML}?}


\begin{columns}[b]
\begin{column}{5.5cm}
\hspace{-2cm}\begin{block}{}
{\footnotesize
\begin{semiverbatim}
<HTML>
<HEAD>
<TITLE>CSS Example</TITLE>
</HEAD>

<BODY>
<H1>Title of web page</H1>

<P>A first paragraph...</P>

<P>Another paragraph, with an
<A HREF=".">anchor
(reference)</A>...</P>
</BODY>

</HTML>




\end{semiverbatim}
}
\end{block}
\end{column}

\begin{column}{3cm}
\vspace{0.25cm}
{\hspace*{-1cm}\includegraphics[width=5.5cm]{Figures/css-example0.png}}
\vspace{2cm}
\end{column}


\end{columns}


\NOTE{HTML = Hypertext Meta Language, a subset/derivative of SGML for
describing the contents of web pages.}

\end{frame}


\FRAME{Si oui, alors vous avez d\'ej\`a utilis\'e un DSL\\=
\alert{\emph{Domain Specific Language}}\\= Langage Sp\'ecifique au Domaine}



%%%%%%%%%%%%%%%%%%%%%%%%%%%%%%
\section{Qu'est-ce qu'un <<Langage Sp\'ecifique au Domaine>>?}
%--------------------------------------------
\begin{frame}
\frametitle{Qu'est-ce qu'un \emph{Langage Sp\'ecifique au Domaine} (DSL)?}


\large\Alert{blue}{\bf Domain-specific language}:

\begin{quote}

A computer \alert{programming language} of
\Alert{orange}{limited expres\-siveness} focused on a
\Alert{magenta}{particular domain}.

\bigskip

\rSource{M. Fowler, 2011}


\end{quote}

\bigskip
\vfill
\bigskip

\begin{columns}[b]
\begin{column}{5cm}
{\hspace{5cm}\includegraphics[width=3cm]{Figures/livre-fowler}}
\end{column}

\begin{column}{4cm}
%{\hspace{5cm}\includegraphics[width=3cm]{Figures/fowler}}
\end{column}

\end{columns}


\NOTE{Programming language \Implique\ A language used by humans to
program computers}

\NOTE{Limited expressiveness \Implique\ Not \emph{turing-complete},
not a GPL (General Purpose Language) like C, Java, Ruby, etc.}

\NOTE{Particular domain \Implique\ Tied to a specific application
domain---a specific problem area---so it (generally) uses concepts
from that application domain.}

\end{frame}


%--------------------------------------------
\begin{frame}
\frametitle{Qu'est-ce qu'un \emph{Langage Sp\'ecifique au Domaine} (DSL)?}


\large\Alert{blue}{\bf Domain-specific language}:

\begin{quote}

A language offering \alert{expressive power focused on a
	particular problem domain}, such as a specific class of
	applications or \Alert{dgreen}{application aspect}.

\bigskip

\rSource{K. Czarnecki, 2005}

\end{quote}

\NOTE{Expressiveness is limited with respect to Turing-complete/GPL.
However, in the domain for which it is defined, it is \alert{very
expressive}: ``DSLs trade generality for expressiveness in a limited
domain.'' \cite{MernikHeeSlo05} }


\end{frame}



%--------------------------------------------
\begin{frame}
\frametitle{Qu'est-ce qu'un \emph{Langage Sp\'ecifique au Domaine} (DSL)?}

\large\Alert{blue}{\bf Domain-specific language}:

\begin{quote}
A ``little language'', commonly known as a ``domain-specific
language'' (DSL), is a language that is \alert{expressive uniquely
over the specific features of programs in a given problem domain.}

\bigskip

\rSource{A. Yepez, 2010}
\end{quote}

\NOTEvide

\end{frame}


%------------------------------------------------
\begin{frame}
\frametitle{Un DSL vise \`a r\'eduire le <<foss\'e
s\'emantique>>}
\framesubtitle{Le foss\'e entre id\'ees de l'usager et leur expression
dans un programme}

\hspace*{0.4cm}\includegraphics[width=8.5cm]{Figures/semantic-gap}

\NOTE{\ }

\end{frame}

%------------------------------------------------
\begin{frame}
\frametitle{Un DSL vise \`a r\'eduire le <<foss\'e
s\'emantique>>}
\framesubtitle{Le foss\'e entre id\'ees de l'usager et leur expression
dans un programme}

\hspace*{0.4cm}\includegraphics[width=8.5cm]{Figures/semantic-gap-2}

\NOTE{\ }

\end{frame}



%------------------------------------------------
\begin{frame}
\frametitle{Un DSL utilise le langage du domaine d'application}


\vspace{-0.3cm}

\includegraphics[width=11cm]{Figures/problem-DSL-solution}

\bigskip

\rsource{``DSLs in Action'', D.~Ghosh, 2011}


\NOTE{Problem domain: Process, entities, constraints part of the
business being analyzed}

\NOTE{Solution domain: Implementation aspects involving computer
science and software engineering tools and techniques}

\NOTE{Problem domain[:] ``all information that defines the problem and
constrains the solution (the constraints being part of the problem).
It includes the goals that the problem owner wishes to achieve, the
context within which the problem exists, and all rules that define
essential functions or other aspects of any solution product.''

Source~: {\url{http://c2.com/cgi/wiki?ProblemDomain}}
}

\NOTE{Common vocabulary et \emph{ubiquituous language} de Evans!}
\end{frame}


%%%%%%%%%%%%%%%%%%%%%%%%%%%%%%%%%%%%%%%%%%%%%%%%%%%%%%%%%%%%%%%%%%%%%%%%%%%%%%
%%%%%%%%%%%%%%%%%%%%%%%%%%%%%%%%%%%%%%%%%%%%%%%%%%%%%%%%%%%%%%%%%%%%%%%%%%%%%%

\Subsection{Quelques exemples de DSLs}
\input{dsl_exemples}


%%%%%%%%%%%%%%%%%%%%%%%%%%%%%%%%%%%%%%%%%%%%%%%%%%%%%%%%%%%%%%%%%%%%%%%%%%%%%%
%%%%%%%%%%%%%%%%%%%%%%%%%%%%%%%%%%%%%%%%%%%%%%%%%%%%%%%%%%%%%%%%%%%%%%%%%%%%%%

\Subsection{Pourquoi utiliser des DSLs?}


%------------------------------------------------
\begin{frame}
\frametitle{Un DSL permet un meilleur niveau d'abstraction}

\begin{columns}[b]


\begin{column}{7cm}
{\hspace{-0.5cm}\includegraphics[width=7.2cm]{Figures/abstraction-2}}
\end{column}

\begin{column}{3cm}

\source{\url{geek-and-poke.com}}
\end{column}
\end{columns}

\NOTE{A DSL provides high level abstractions that match the level of
abstraction of the application domain.}

\NOTE{Having a well-designed DSL often leads to a better separation
between the domain rules and the technical code implementing those
rules:  \alert{technical implementation details should not be present} in
the DSL script.}

\end{frame}

%------------------------------------------------
\begin{frame}
\frametitle{Un DSL est un langage plus expressif}

{\hspace{-0.5cm}\includegraphics[width=7.5cm]{Figures/expressiveness-1}}

\NOTE{A DSL provides a very expressive language, again because it
matches the concepts of the application domain}

\NOTE{High level abstractions combined with expressiveness lead to
\alert{conciseness}, thus to better \alert{productivity}}


\end{frame}

%------------------------------------------------
\begin{frame}
\frametitle{Un DSL facilite la comunication avec les experts du domaine}


\vspace{-0.2cm}

{\hspace{-0.5cm}\includegraphics[width=7.4cm]{Figures/expert-1}}

\NOTE{Expressiveness with respect to the application domain also means
that a DSL script should be easier to read than a script in a regular
programming language. This should thus \alert{improve the communication with
the domain experts}.}

\NOTE{In fact, a DSL allows domain experts to be involved in the
specification of the business rules. Requiring those domain experts to
be able to \alert{write} using the DSL might be a bit ``too much to
ask'', but at least they should be able to easily \alert{read} them,
and possibly (?) modify them.}

\end{frame}



%------------------------------------------------
\begin{frame}
\frametitle{D\'esavantages des DSLs}

\begin{block}{}
$\bullet$ Concevoir un langage est difficile: 

\hspace*{1cm} \underline{\alert{\Large D}}SL \Implique\
Connaissance du \alert{\underline{\Large D}omain}e
\end{block}

\vfill

\begin{block}{}
$\bullet$ Un autre langage (nouveau) \`a apprendre
\end{block}


\vfill

\begin{block}{}
$\bullet$ Une autre couche logicielle
\end{block}


\vfill

\NOTE{Another layer \Implique\ There might be performance concerns in
the case of an internal DSL, since such DSL are generally implemented
using scripting (non-compiled) languages. However, as for any other
program and software, optimization should be a concern only when we can know
if/where there is indeed a problem.}

\NOTE{As Don Knuth wrote: ``We should forget about small
efficiencies, say about 97\% of the time: \alert{premature
optimization is the root of all evil}''}

\end{frame}


% SSSSSSSSSSSSSSSSSSSSSSSSSSSSSSSSSSSSSSSSSSSSSSSSSSSSSSSSSSSSSSSSSSSSSSSSSSSSSSSS
\Subsection{Diff\'erentes sortes de DSLs}

%------------------------------------------------
\begin{frame}
\frametitle{DSL externe vs.\ DSL interne:\\Caract\'erisation de Fowler}

\

\begin{block}{DSL externe}
\begin{quote}
``An \alert{external DSL} is a completely separate language, for which
you [need] a full parser.''

\medskip

\rSource{Fowler, 2009}
\end{quote}
\end{block}

\vfill

\begin{block}{DSL interne}
\begin{quote}
An \alert{internal DSL} is an
idiomatic way of using a general-purpose language.''

\medskip

\rSource{Fowler, 2009}

\end{quote}
\end{block}

\vfill

\NOTE{\ }

\end{frame}

%--------------------------------------------
\begin{frame}[plain]
\includegraphics[width=11cm]{Figures/internal-vs-external-0}

\NOTE{External vs. internal:
\begin{itemize}

\item External: the language is distinct from the language used to
develop the application, so an independent parser/interpreter must be
built---using the usual/typical tools (e.g., lexer/parser generator
such as lex/yacc, flex/bison, antlr, etc.)

\item Internal: is a ``sublanguage'' of the language used to develop
the application, i.e., it ``uses'' the infrastructure of an existing
programming language (called the host language). In other words, it is
implemented ``\alert{on top of}'' an existing programming language, so
it must obey the constraints (syntax, semantics) of the host language.
\end{itemize}
}

\end{frame}

%--------------------------------------------
\begin{frame}[plain]
\includegraphics[width=11cm]{Figures/internal-vs-external-1}
\NOTE{\ }
\end{frame}



%------------------------------------------------
\begin{frame}
\frametitle{DSL externe vs.\ DSL interne~:\\Exemples}


\begin{columns}
\begin{column}{5cm}
\begin{block}{{DSL Externes}}
\begin{itemize}
\item HTML
\item CSS
\item Graphviz
\item Make
\item Ant
\end{itemize}
\end{block}
\end{column}

\begin{column}{5cm}
\begin{block}{{DSL Internes}}
\begin{itemize}
\item Rake
\item Gli
\item RSpec
\item Scripts de correction Oto
\end{itemize}
\end{block}
\end{column}
\end{columns}

\vfill

\NOTE{\ }

\end{frame}

%------------------------------------------------
\begin{frame}
\frametitle{DSL autonome vs.\ DSL fragmentaire~:\\Caract\'erisation de Fowler}



\begin{block}{DSL autonome}
\begin{quote}
``[A stand-alone DSL is used through a] DSL script, typically in a
single file, and \alert{it is all DSL}.''
\medskip

\rSource{Fowler, 2009}
\end{quote}
\end{block}

\vfill

\begin{block}{DSL fragmentaire}
\begin{quote}
\item ``[With a fragmentary DSL,] \alert{little bits of DSL} are used
\alert{inside} the host language code. You can think of them as
\alert{enhancing the host language} with additionnal features.''

\medskip

\rSource{Fowler, 2009}

\end{quote}
\end{block}

\vfill


\NOTE{\ }

\end{frame}


%------------------------------------------------
\begin{frame}[plain]
\frametitle{DSL autonome vs.\ DSL fragmentaire}

\includegraphics[width=12cm]{Figures/standalone-vs-fragmentary-1}

\NOTE{Stand-alone: 

\begin{itemize}

\item ``[Used through a] DSL script, typically in a single file, and
\alert{it is all DSL}.''

\item ``In this case, you could/should follow what the DSL is doing
without understanding the host language.''
\end{itemize}
}

\NOTE{Fragmentary:
\begin{itemize}
\item ``\alert{Little bits of DSL} are used \alert{inside} the host language
code. You can think of them as \alert{enhancing the host language}
with additionnal features.''

\item ``In this case, you can't really follow what the DSL is doing
without understanding the host language.''
\end{itemize}
}

\end{frame}


%------------------------------------------------
\begin{frame}
\frametitle{DSL autonome vs.\ DSL fragmentaire~:\\Exemples}

\begin{columns}
\begin{column}{5cm}
\begin{block}{{DSL Autonomes}}
\begin{itemize}
\item HTML
\item CSS
\item graphviz
\item Make
\item Ant
\item Rake
\item Scripts de correction Oto
\end{itemize}	
\end{block}
\end{column}

\begin{column}{5cm}
\begin{block}{{DSL Fragmentaires}}
\begin{itemize}
\item Gli
\item Expressions r\'eguli\`eres
\item SQL enchass\'e
\end{itemize}
\end{block}
\end{column}
\end{columns}

\NOTE{\ }

\end{frame}


%------------------------------------------------
\begin{frame}
\frametitle{DSL applicatif vs. DSL utilitaire~:\\Caract\'erisation de
Voelter}

\begin{block}{\em Application domain DSL}
\em [These] DSLs describe the \alert{core business logic of an application system}
independent of its technical implementation.  These DSLs are intended
\alert{to be used by domain experts, usually non-programmers}. 
\end{block}

\VF

\begin{block}{\em Utility DSL}
\em [These] DSLs [act] simply \Alert{blue}{as utilities for developers}. A
developer, or a small team of developers, creates a small DSL that
\Alert{blue}{automates a specific}, usually \Alert{blue}{well-bounded aspect of
software development}.
\end{block}

\VF

\source{<<DSL Engineering: Designing, Implementing and Using
Domain-Specific Languages>>, Voelter}

\NOTE{Contrainte sur les DSLs applicatifs, puisqu'utilis\'es par des
non-programmeurs~: <<\emph{more stringent requirements regarding
notation, ease of use and tool support.}>>}

\NOTE{Avec un DSL utilitaire~: <<The overall development process is
not based on DSLs, \Alert{blue}{it's a few developers being creative
and simplifying their own lives}.>>}



\end{frame}

%------------------------------------------------
\begin{frame}
\frametitle{Les diff\'erentes cat\'egories~: Exemples}

\begin{block}{}
{\large
\begin{center}
\begin{tabular}{l||p{3cm}|p{3cm}|}
 & {\bf \Alert{maincolor}{Externes}} & {\bf \Alert{maincolor}{Internes}} 
\\\hline\cline{2-3}
{\bf \Alert{maincolor}{Autonomes}} 
   & HTML & \Alert{blue}{Rake}
\\
  & CSS & 
\\
   & \Alert{blue}{Make} & 
\\
   & \Alert{blue}{Ant} & 
\\
   & {Graphviz} & 
\\
   & {\LaTeX} & 
\\
   & \ldots & Scripts Oto
\\\hline
{\bf \Alert{maincolor}{Fragmentaires}} 
   & SQL enchass\'e & \Alert{blue}{Gli}
\\
   & & \Alert{blue}{RSpec}
\\
   & & 
\\\cline{2-3}
\end{tabular}

\end{center}
}
\end{block}

L\'egende~: \Alert{blue}{Utilitaire}

\NOTE{\ }


\end{frame}

%%%%%%%%%%%%%%%%%%%%%%%%%%%%%%%%%%%%%%%%%%%%%%%%%%%%%%%%%%%%%%%%%%%%%%%%%%%%%%
%%%%%%%%%%%%%%%%%%%%%%%%%%%%%%%%%%%%%%%%%%%%%%%%%%%%%%%%%%%%%%%%%%%%%%%%%%%%%%
\section{Mise en oeuvre des DSLs}

\Subsection{Mise en oeuvre des DSLs externes}

%------------------------------------------------
\begin{frame}
\frametitle{Mise en oeuvre <<classique>> d'un DSL externe}

\vfill
{\hspace*{-0.5cm}\includegraphics[width=12cm]{Figures/dsl-externe-2}}

\vfill


\NOTE{\ }

\end{frame}


%------------------------------------------------
\begin{frame}
\frametitle{Mise en oeuvre <<classique>> d'un DSL externe}

\vfill
{\hspace*{-0.5cm}\includegraphics[width=12cm]{Figures/dsl-externe-1}}

\vfill


\NOTE{\ }

\end{frame}



%------------------------------------------------
\begin{frame}
\frametitle{Mise en oeuvre <<classique>> d'un DSL externe}

\vfill
{\hspace*{-0.5cm}\includegraphics[width=12cm]{Figures/dsl-externe-0}}

\vfill


\NOTE{\ }

\end{frame}

%------------------------------------------------
\begin{frame}
\frametitle{Mise en oeuvre <<classique>> d'un DSL externe}

\vfill
{\hspace*{-0.5cm}\includegraphics[width=12cm]{Figures/dsl-externe-00}}

\vfill


\NOTE{\ }

\end{frame}

%------------------------------------------------
\begin{frame}
\frametitle{Plus r\'ecemment, on a vu apparaitre des \emph{language workbenches}}

\begin{quote}
A \alert{Language Workbench} (LWB) is a development toolset that facilitates
the development and editing of domain specific languages (DSLs). 

\bigskip

\source{\url{http://searchsoftwarequality.techtarget.com/definition/Language-Workbench}}
\end{quote}

\vfill
\NOTE{\ }

\end{frame}

%------------------------------------------------
\begin{frame}
\frametitle{Plus r\'ecemment, on a vu apparaitre des \emph{language workbenches}}

\vspace*{-0.6cm}

\begin{center}
{\includegraphics[width=8.5cm]{Figures/language-workbench-2}}
\end{center}

\vfill

\NOTE{\source{\url{http://ptgmedia.pearsoncmg.com/images/RCexcerpt_fowler_dsl1/elementLinks/01fig07.jpg}}}

\end{frame}

%------------------------------------------------
\begin{frame}
\frametitle{Plus r\'ecemment, on a vu apparaitre des \emph{language workbenches}}

{\hspace*{-0.5cm}\includegraphics[width=12cm]{Figures/language-workbench}}

\vfill

\NOTE{\source{\url{http://www.martinfowler.com/articles/languageWorkbench.html}}}

\end{frame}

%%%%%%%%%%%%%%%%%%%%%%%%%%%%%%%%%%%%%%%%%%%%%%%%%%%%%%%%%%%%%%%%%%%%%%%%%%%%%%
%%%%%%%%%%%%%%%%%%%%%%%%%%%%%%%%%%%%%%%%%%%%%%%%%%%%%%%%%%%%%%%%%%%%%%%%%%%%%%

\Subsection{Mise en oeuvre des DSLs internes}

%------------------------------------------------
\begin{frame}
\frametitle{Techniques de mises en oeuvre de DSLs internes}

\huge Voir sections suivantes\ldots


\NOTE{\ }

\end{frame}


%%%%%%%%%%%%%%%%%%%%%%%%%%%%%%
\section{Des DSL internes pour sp\'ecifier des documents <<\`a la XML>>}
\input{dsl_exemple-documents}
\input{dsl_exemple-documents-java}

%%%%%%%%%%%%%%%%%%%%%%%%%%%%%%
\section{Un exemple~: Un \emph{gem} pour sp\'ecifier des suites de commandes}
\input{dsl_exemple-gli}

%%%%%%%%%%%%%%%%%%%%%%%%%%%%%%
\section{Un exemple C++~: Configuration de programmes parall\`eles} \input{dsl_exemple-fastflow}

%%%%%%%%%%%%%%%%%%%%%%%%%%%%%%
\section{Conclusion}
%------------------------------------------------
\begin{frame}
\frametitle{Ruby est un langage puissant et expressif}


\begin{block}{}
\em Ruby is ``A dynamic, open source programming language with {a focus on
simplicity and productivity}.''

\medskip

\rSource{\url{https://www.ruby-lang.org/en/}}
\end{block}

\vfill

\pause


\begin{block}{}
\em Ruby \alert{is ``made  for developer happiness''!}

\medskip

\rSource{Y. Matsumoto, creator of Ruby}

\vfill


\end{block}


\NOTE{As for \alert{developer happiness}, that is how I felt when I
discovered Ruby. I did a couple of years of Lisp programming (during
my PhD thesis). Then, I did a couple of years of programming in Perl.
I really enjoyed how powerful Perl was, with its dynamic typing and
powerful string processing, but the programs were awful, ugly,
difficult to read once they had been written.}

\NOTE{And of course, compared to Java, I really enjoyed how much more
concise Ruby programs were.}

\end{frame}

%------------------------------------------------
\begin{frame}
\frametitle{Ruby facilite la d\'efinition de DSLs internes expressifs}

\begin{block}{Syntaxe flexible}
\end{block}

\VF

\begin{block}{Argument par mots-cl\'es}
\end{block}

\VF

\begin{block}{Blocs et fermetures}
\end{block}

\VF

\begin{block}{M\'etaprogrammation dynamique}
\end{block}

\VF




\NOTE{\ }

\end{frame}


%------------------------------------------------
\begin{frame}
\frametitle{Mais on peut d\'efinir des DSLs internes dans n'importe quel langage\ldots\ m\^eme Java}


\begin{block}{Constructeurs et chainage de m\'ethodes}
\end{block}

\VF

\begin{block}{Lambda-expressions}
\end{block}

\VF

\begin{block}{Annotations}
\end{block}

\VF

\NOTE{\ }

\end{frame}


%------------------------------------------------
\begin{frame}
\frametitle{Les DSLs peuvent simplifier l'expression de solutions \`a
de nombreux probl\`emes}

% \begin{block}{}
% \begin{itemize}
% \item \ldots
% \end{itemize}
% \end{block}

% \VF

\begin{block}{La seule limite\only<1>{\ldots}\only<2-3>{ pour le {Projet~\#3}}}

\only<1-2>{\huge\alert{\ \ \ }}
\only<3>{\huge\alert{Votre imagination!}}

\end{block}

\NOTE{\ }

\end{frame}




%SSSSSSSSSSSSSSSSSSSSSSSSSSSSSSSSSSSSSSSSSSSSSSSSSSSSSSSSSSSSSSS


%------------
\begin{frame}
\frametitle{Pour des informations additionnelles sur les DSLs}


\begin{columns}[b]

\begin{column}{5.5cm}
{
\includegraphics[width=5cm]{Figures/livre-fowler}
}
\end{column}

\begin{column}{5.5cm}
{
\includegraphics[width=5.5cm]{Figures/livre-ghosh}
}

\end{column}

\end{columns}

\NOTE{The book by Fowler is organized in terms of patterns for
defining DSL---internal vs. external ones.}

\NOTE{The book by Ghosh is organized more in terms of
language-specific techniques---Ruby, Groovy, Scala, closure.}


\end{frame}

%------------
\begin{frame}
\frametitle{Pour des informations additionnelles sur les DSLs}

\nocite{Fowler11}
\nocite{Ghosh11}
\nocite{Bentley86.2}
%\nocite{Czarnecki05}
\nocite{MernikHeeSlo05}
\nocite{ArkinTek14}
\nocite{Fowler09}
\nocite{Voelter13}

\bibliographystyle{alpha}

\vspace{-0.25cm}

{\tiny
\bibliography{%
biblio/design%
,%
biblio/varia%
,%
biblio/comp%
,%
biblio/arch+pp%
}
}

\NOTEvide

\end{frame}



\end{document}

