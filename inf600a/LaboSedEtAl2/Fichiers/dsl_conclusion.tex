%------------------------------------------------
\begin{frame}
\frametitle{Ruby est un langage puissant et expressif}


\begin{block}{}
\em Ruby is ``A dynamic, open source programming language with {a focus on
simplicity and productivity}.''

\medskip

\rSource{\url{https://www.ruby-lang.org/en/}}
\end{block}

\vfill

\pause


\begin{block}{}
\em Ruby \alert{is ``made  for developer happiness''!}

\medskip

\rSource{Y. Matsumoto, creator of Ruby}

\vfill


\end{block}


\NOTE{As for \alert{developer happiness}, that is how I felt when I
discovered Ruby. I did a couple of years of Lisp programming (during
my PhD thesis). Then, I did a couple of years of programming in Perl.
I really enjoyed how powerful Perl was, with its dynamic typing and
powerful string processing, but the programs were awful, ugly,
difficult to read once they had been written.}

\NOTE{And of course, compared to Java, I really enjoyed how much more
concise Ruby programs were.}

\end{frame}

%------------------------------------------------
\begin{frame}
\frametitle{Ruby facilite la d\'efinition de DSLs internes expressifs}

\begin{block}{Syntaxe flexible}
\end{block}

\VF

\begin{block}{Argument par mots-cl\'es}
\end{block}

\VF

\begin{block}{Blocs et fermetures}
\end{block}

\VF

\begin{block}{M\'etaprogrammation dynamique}
\end{block}

\VF




\NOTE{\ }

\end{frame}


%------------------------------------------------
\begin{frame}
\frametitle{Mais on peut d\'efinir des DSLs internes dans n'importe quel langage\ldots\ m\^eme Java}


\begin{block}{Constructeurs et chainage de m\'ethodes}
\end{block}

\VF

\begin{block}{Lambda-expressions}
\end{block}

\VF

\begin{block}{Annotations}
\end{block}

\VF

\NOTE{\ }

\end{frame}


%------------------------------------------------
\begin{frame}
\frametitle{Les DSLs peuvent simplifier l'expression de solutions \`a
de nombreux probl\`emes}

% \begin{block}{}
% \begin{itemize}
% \item \ldots
% \end{itemize}
% \end{block}

% \VF

\begin{block}{La seule limite\only<1>{\ldots}\only<2-3>{ pour le {Projet~\#3}}}

\only<1-2>{\huge\alert{\ \ \ }}
\only<3>{\huge\alert{Votre imagination!}}

\end{block}

\NOTE{\ }

\end{frame}




%SSSSSSSSSSSSSSSSSSSSSSSSSSSSSSSSSSSSSSSSSSSSSSSSSSSSSSSSSSSSSSS


%------------
\begin{frame}
\frametitle{Pour des informations additionnelles sur les DSLs}


\begin{columns}[b]

\begin{column}{5.5cm}
{
\includegraphics[width=5cm]{Figures/livre-fowler}
}
\end{column}

\begin{column}{5.5cm}
{
\includegraphics[width=5.5cm]{Figures/livre-ghosh}
}

\end{column}

\end{columns}

\NOTE{The book by Fowler is organized in terms of patterns for
defining DSL---internal vs. external ones.}

\NOTE{The book by Ghosh is organized more in terms of
language-specific techniques---Ruby, Groovy, Scala, closure.}


\end{frame}

%------------
\begin{frame}
\frametitle{Pour des informations additionnelles sur les DSLs}

\nocite{Fowler11}
\nocite{Ghosh11}
\nocite{Bentley86.2}
%\nocite{Czarnecki05}
\nocite{MernikHeeSlo05}
\nocite{ArkinTek14}
\nocite{Fowler09}
\nocite{Voelter13}

\bibliographystyle{alpha}

\vspace{-0.25cm}

{\tiny
\bibliography{%
biblio/design%
,%
biblio/varia%
,%
biblio/comp%
,%
biblio/arch+pp%
}
}

\NOTEvide

\end{frame}

